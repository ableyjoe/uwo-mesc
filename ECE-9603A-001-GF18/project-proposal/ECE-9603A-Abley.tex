\documentclass[9pt,letterpaper,twocolumn]{scrartcl}

\usepackage{cite}
\usepackage{url}

\begin{document}

\titlehead{{\Large Western University \hfill ECE 9603\\}
  Electrical and Computer Engineering \hfill Fall 2018}
\subject{Project Proposal}
\title{Predicting the Origin of DNS Traffic Without Reference to Client Source Address}
\author{Joe Abley, \texttt{jabley@uwo.ca}}

\maketitle

% \begin{abstract}
% We propose a predictive system which is able to identify the
% originating system responsible for stateless Domain Name System
% (DNS) traffic received at a authoritative DNS server without
% reference to the source address. The ability to predict whether
% particular DNS query traffic received at an authoritative server
% is legitimately sourced from a particular client system is useful
% in identifying some classes of malicious traffic in production DNS
% systems.
% \end{abstract}

\section{Problem Description}
The Domain Name System (DNS) includes a wire protocol with which
structured requests and responses are exchanged over a network. The
DNS protocol is specified and widely used using the UDP transport
protocol.  Since UDP is stateless there is no way for the receiver
of a DNS request sent over UDP to verify the legitimacy of a source
address.  A consequence of this is that DNS servers are frequently
used as amplifiers in reflection attacks\cite{RFC5358}. Although
some such attacks are trivially identified, for example by Query
Type (QTYPE), many are more difficult.  By choosing query parameters
that match legitimate, real-world use of the DNS, the attacks may
seem impractical to block without collateral damage. This is
especially true of amplification attacks against DNS resolvers.

The goal of this project is to construct a classifier that can
distinguish between a stream of DNS queries received from one
real-world client and a stream received from another without reference
to source address, based on training sets derived from queries whose
origin is accurately known.

\section{Data Sets}
Afilias operates authoritative DNS servers for the INFO top-level
domain, as well as many others. These servers are made available
to the Internet through commodity transit arrangements from multiple
locations using anycast\cite{RFC4786}, as well as through so-called
Private Network Interconnects (PNIs). The origin of queries received
over PNIs can be said to be known with high accuracy; the origin
of queries received over the Internet, in contrast, cannot.

Complete sets of query data received over both trusted and untrusted
paths are available from Afilias for use in this project in the
form of packet captures in PCAP format\footnote{PCAP, named after
the C library \texttt{libpcap}, is the file format used by the
\texttt{tcpdump} utility.}.

Time-series data sets are constructed from query data by identifying
features that distinguish different queries and creating a vector
of metrics for each class of queries along a time axis calculated
within regular time intervals. Examples of features are the proportion
of queries with \texttt{QTYPE=MX} or the number of labels in the
Query Name; corresponding coordinates in the time-series vector
might be the mean, standard deviation and 90th percentile of each
such metric. This method has been used successfully by Castro and Qiao to
build similar classifiers based on DNS query data\cite{CastroQiao}.

The identification of a set of features that allows a useful
classifier to be trained is the primary desired result of this
project. Choice of algorithm, hyperparameters, etc is expected to
be determined automatically using a suitable machine learning toolkit
such as \texttt{auto-sklearn}\cite{NIPS2015_5872}.

\section{Real-World Applications}
Traffic received over the Internet that purports to be from a
particular client but which is classified otherwise can be isolated
itself be subject to classification. Being able to classify known-bad
traffic from packet captures after the fact is useful in forensic
analysis of strange traffic patterns (e.g. noticeable spikes in
traffic volume) and potentially also for use in real-time if the
matching criteria can be distilled down into something with low
cost to execute, like a Bloom filter.

Finding traffic from known clients arriving from an unexpected
direction also has the potential to inform operational management
systems, e.g. causing an alarm in the network operations centre
that could prompt an escalation. An example of such a client might
be Google Public DNS, who are relied upon by a large number of
end-users.

\bibliographystyle{abbrvurl}
\bibliography{ECE-9603A-Abley}
\end{document}
